\newpage
%%%%%%%%%%%%%%%%%%%%%%%%%%%%%%%%%%%%%%%%%%%%%%%%%%%%%%%%%%%%%%%%%%%%%%%%
\section{Introduction}
\label{sec:int}
The production of specific charm hadrons has
been measured in different regimes and environments: in 
\epem collisions at $B$-factories~\cite{Bortoletto:1988kw,
Avery:1990bc,Albrecht:1991ss,Albrecht:1991pa,
Albrecht:1988an,Aubert:2002ue,
Seuster:2005tr,Aubert:2006cp}  and in $Z$
decays~\cite{Alexander:1996wy,Ackerstaff:1997ki,Barate:1999bg, 
Abreu:1999vw,Abreu:1999vx}, in \ep collisions in photoproduction 
(PHP)~\cite{Chekanov:2005mm,Abramowicz:2013eja},  deep inelastic 
scattering (DIS)~\cite{Chekanov:2007ch,
Abramowicz:2010aa,Aktas:2004ka} and in
$pp$ collisions~\cite{Aaij:2013mga}.

The fragmentation process is soft and hence can not be calculated with
the techniques of perturbative QCD (pQCD).
%
Therefore, these measurements 
are a necessary ingredient for any QCD prediction of 
charm-hadron production.
%
In this context, it is important to validate the hypothesis 
that fragmentation 
fractions are universal, i.e.\ independent of the hard production 
mechanism. Thus, once precisely measured in one experiment, they can 
be applied in any reaction.
%
Another important check is that the sum of fragmentation fractions of 
all known weakly decaying charm hadrons is equal to unity, thus checking if all 
weakly decaying states are known.

To achieve these goals, a comparison of fragmentation-fraction 
measurements obtained in different production regimes 
is performed using a combination of individual measurements.
%
Due to independent data sets and different detector types and 
constructions, the experimental statistical and systematic uncertainties 
in most cases can be treated as uncorrelated between measurements.
%
However, a careful treatment of correlated uncertainties due to common 
usage of
branching ratios and theory inputs is essential, as in many measurements 
these are one of the leading uncertainty sources.
%
In the past several combinations of fragmentation-fraction data were 
performed with fewer inputs: the summary of the charm fragmentation 
fractions in \epem at the $Z$ resonance~\cite{PDG2014}, the combination 
of \epem measurements~\cite{Gladilin:1999pj,Gladilin:2014tba} as well as 
the combination of \epem and $e^{\pm}p$ 
measurements~\cite{Lohrmann:2011np}.
%
Compared to those, the present analysis extends to a larger set of 
measurements, in particular the final measurement in PHP by the ZEUS 
experiment at HERA~\cite{Abramowicz:2013eja},  the $pp$ measurements 
from LHCb~\cite{Aaij:2013mga}, and the $\Lambda_c^{+}$ measurements from 
the BABAR experiment~\cite{Aubert:2006cp}. It uses the up-to-date 
branching-ratio values~\cite{PDG2014,Zupanc:2013iki,Ablikim:2014mww,
Aubert:2005ik}, treats correlations of 
branching-ratio uncertainties and recent theory predictions with 
reduced uncertainties~\cite{Chetyrkin:2000zk,Freitas:2014hra} as input.

%%%%%%%%%%%%%%%%%%%%%%%%%%%%%%%%%%%%%%%%%%%%%%%%%%%%%%%%%%%%%%%%%%%%%%%%
\section{Combination of individual measurements}
\label{sec:comb}
\subsection{Update of input measurements to recent branching ratios}
\label{sec:pdgupdate}
To make separate inputs consistent, the original measurements are 
corrected to the same up-to-date world averages of branching ratios of 
the charm-hadron decays, summarised in Tab.~\ref{tab:BIGPDG}. Most of 
the values were taken from Ref.~\cite{PDG2014}. Exceptions are made
 for the  branching ratios of $\Lambda_c^{+}$ and $D^{*0}$ decays.
%
The branching ratio ${\cal B}(\Lambda_c^{+} \rightarrow p K\pi)$ value 
is taken from the recent measurement of BELLE~\cite{Zupanc:2013iki} and  
is much more precise than the current world average~\cite{PDG2014}. 
%
The branching ratios of the other $\Lambda_{c}^{+}$ decay channels were 
calculated from their ratios to  the ${\cal B}(\Lambda_c^{+}\rightarrow
p K\pi)$ measured elsewhere~\cite{Link:2005ut,Avery:1993vj,Alam:1998nb}. 
Therefore, the new precise BELLE data~\cite{Zupanc:2013iki} allowed 
the uncertainties of these to be reduced as well.
%
The ${\cal B}(D^{*0} \rightarrow D^{0}\pi^0)$ was calculated from the 
two most precise measurements~\cite{Ablikim:2014mww,Aubert:2005ik} of 
${\cal B}(D^{*0} \rightarrow D^{0}\pi^0)/{\cal B}(D^{*0} 
\rightarrow D^{0}\gamma)$ assuming ${\cal B}(D^{*0} \rightarrow 
D^{0}\pi^0)+{\cal B}(D^{*0} \rightarrow D^{0}\gamma)=1$.
\tabBIGPDG

\subsection{Calculation of the fragmentation fractions}
\label{sec:comb::ffdef}
In this paper the charm-quark  fragmentation fraction to a specific 
hadron is defined as the ratio of the production cross-section over the 
production cross-section of the charm quark:
%-----------------------------------------------------------------------
\begin{equation}\label{eq:ffwithcharmcs}
f(c \rightarrow H) = {\sigma(H)}/{\sigma(c)}. 
\end{equation}
%-----------------------------------------------------------------------
The Standard Model makes precise predictions for the total charm 
cross-section in $e^+e^-$ collisions, therefore, for those processes it 
is possible to calculate $f(c \rightarrow H)$ according to 
\eq{ffwithcharmcs}.
%
Sufficiently precise predictions for the charm-quark production 
in $pp$ and $e^{\pm}p$ collisions are not available.
%
However, making an assumption that the sum of charm-quark fragmentation 
fractions to all known weakly decaying charm hadrons is unity, the 
charm-quark fragmentation fraction to a specific hadron can be 
calculated as the ratio of the hadron-production cross-section over the 
sum of cross-sections of all known weakly decaying  charm hadrons
%-----------------------------------------------------------------------
\begin{equation}\label{eq:ffwithsumh}
f(c\rightarrow H)={\sigma(H)}/{\Sigma_{i}\sigma(H_{i})}. 
\end{equation}
%-----------------------------------------------------------------------

To obtain the charm-quark fragmentation fractions according to 
\eq{ffwithsumh}, in addition to the production cross-sections of $D$ 
mesons and $\Lambda_c^{+}$, it is necessary to know the cross-sections 
of the weakly decaying $\Xi_c$ and $\Omega_c$ states.
%
Those states are poorly studied, therefore as in 
Ref.~\cite{Alexander:1996wy} it is assumed that ratios of fragmentation 
fractions of charm and strange quarks into the corresponding baryons are 
similar,
 $f(c\rightarrow \Xi_c^+)/f(c\rightarrow \Lambda_c)=
  f(c\rightarrow \Xi_c^0)/f(c\rightarrow \Lambda_c)=
  f(s\rightarrow \Xi^{-})/f(s\rightarrow \Lambda^0)$
and  
 $f(c\rightarrow \Omega_c)/f(c\rightarrow \Lambda_c)=
  f(s\rightarrow \Omega)/f(s\rightarrow \Lambda^0).
  $
%
In this approach the sum of the production cross-sections of these states can be estimated
as
\begin{equation}
\sigma(\Xi_c^+)+\sigma(\Xi_c^0)+\sigma(\Omega_c)=
2\frac{f(s\rightarrow \Xi^{-})}{f(s\rightarrow \Lambda^0)}\sigma(\Lambda_{c}^{+})
+\frac{f(s\rightarrow \Omega)}{f(s\rightarrow \Lambda^0)}\sigma(\Lambda_{c}^{+})
=\lambda\sigma(\Lambda_{c}^{+}),
\end{equation}
where we define
\begin{equation}
\lambda=f(s\rightarrow \Omega)/f(s\rightarrow \Lambda^0)+2f(s\rightarrow 
 \Xi^{-})/f(s\rightarrow \Lambda^0)=\EXTRABARYONS.
\end{equation}
%-----------------------------------------------------------------------
%where $\lambda f(c\rightarrow \Lambda_{c}^{+})$ is  an estimate of the 
%weakly decaying heavy baryonic states. 
The  value of $\lambda$ is 
calculated using the most precise set of $s$ quark fragmentation 
fractions
$f(s\rightarrow \Xi^{-})=0.0016\pm 0.0003$,
$f(s\rightarrow \Omega)=0.0258\pm 0.0010$ and 
$f(s\rightarrow \Lambda^0)=0.3915\pm 0.0065$ from Ref.~\cite{PDG2014} 
obtained at LEP.
Hereby, the sum of production cross-sections of all weakly decaying 
states is
%-----------------------------------------------------------------------
\begin{equation}
\begin{split}
\label{eq:lambda}
\Sigma_{i}\sigma(H_{i})=\sigma(D^0)+\sigma(D^+)+\sigma(D_s^+)+
\sigma(\Lambda_{c}^{+})+\lambda \sigma(\Lambda_{c}^{+}).
\end{split}
\end{equation}

The fragmentation fractions calculated according to \eq{ffwithcharmcs} 
for the \epem collisions and $Z$ decays 
allow an independent check that
%-----------------------------------------------------------------------
\begin{equation}\label{eq:ffsumisone}
S=f(c\rightarrow D^0)+f(c\rightarrow D^+)+f(c\rightarrow D_s^+)+
f(c\rightarrow \Lambda_{c}^{+})+\lambda f(c\rightarrow \Lambda_{c}^{+})
\end{equation}
%-----------------------------------------------------------------------
is close to unity with sufficient accuracy.
%
\subsection{Combination procedure}
\label{sec:comb:proc}
The combination of the measurements used in the present analysis is 
based on numerical $\chi^2$ minimisation with respect to observables of 
interest.
The numerical minimisation was performed with the MINUIT 
package~\cite{minuit} and the procedure for calculation of $\chi^2$ 
itself is outlined below.
%

For a set of $m$ measurements and corresponding expectation values
 calculated 
from $n$ parameters, a column-vector of the residuals $R(1\times m)$ is 
calculated as a difference of a measurement and the corresponding 
expectation.  The covariance matrix, $V(m \times m)$  is calculated  as 
%-----------------------------------------------------------------------
$$V_{ij}=U^2_{i}\delta_{ij}+\sum_{k}C_{j,k}C_{i,k},$$
%-----------------------------------------------------------------------
where $U_{i}$ stands for an uncorrelated uncertainty of $i$-th residual, 
$C_{i,k}$ stands for the correlated uncertainty of source $k$ of the 
$i$-th measurement and the sum runs over all sources of correlated 
uncertainties.
%
The $\chi^2$ is then calculated as 
%-----------------------------------------------------------------------
$$\chi^2=R^{T}V^{-1}R.$$
%-----------------------------------------------------------------------

The correlated uncertainties
are treated multiplicatively in the
construction of the covariance matrix, i.e.\ the relative uncertainties 
are used to scale the corresponding expectation values 
instead of the measurement.
%
This avoids the bias for normalisation uncertainties, such as branching 
ratio uncertainties, which are the main correlated uncertainties 
considered in the presented analysis.
%
The statistical and uncorrelated systematic uncertainties are treated 
additively.
%
Data sets and their systematic uncertainties are  assumed to be 
independent between experiments.
%
In addition, most of the measurements do not contain the information 
about a potential correlation between cross-section values for different 
charm hadrons.
%
Therefore, in the following all experimental uncertainties  are treated 
as uncorrelated, unless otherwise stated. Uncertainties on the combined 
values of the fragmentation fractions are determined using the Hessian 
method with the criterion $\Delta\chi^2=1$\footnote{
%
As an illustrative example, when
for a given combination set-up the inputs are the
$m$ cross-section measurements. 
%
These data define the $m \times m$ covariance matrix $V$ with the 
experimental statistical and systematic uncertainties contributing 
to the diagonal elements 
and the correlated uncertainties setting the off-diagonal elements 
and contributing to the total uncertainties on the diagonal.
%
The correlated uncertainties considered are those related to
$\lambda$ in \eq{lambda} and branching ratios.
%
The residuals are obtained subtracting from the measurements 
cross-section expectations calculated from $n$ free parameters
in the fit, which could be fragmentation fractions, 
total charm cross-sections, kinematic factors, etc.
%
The details of this calculation are outlined in the each section.
%
With all these components at hand the $\chi^2$ can be evaluated 
and iteratively numerically minimised with respect to the 
free parameters.}.

%
The evaluated total uncertainties on the free parameters comprise 
experimental, branching ratio and  uncertainties of the $\lambda$ 
parameter.

The combination of all the measurements  is obtained imposing the 
normalisation constraint on the sum of all ground state hadrons by 
adding an additional ``measurement'' of $S$ calculated from 
\eq{ffsumisone} with an uncertainty on $\lambda$ and the  corresponding 
prediction $S = 1$.
%
In order to keep the main result with the normalisation constraint as 
model independent as possible, no theory inputs on the charm 
cross-section are used in  such  a combination, and the fragmentation 
fractions are calculated according to
  \eq{lambda}.
%
For the same reason, any measurements that require theoretical inputs 
for conversion into cross-sections or fragmentation fractions and do not 
have these inputs in the original publications are also excluded from 
the main combination.
%
However, such data are included in a more constrained 
combination.
%
In the following,  treatment of such measurements will be discussed 
case-by-case in the relevant sections.

The quantities commonly used as Monte Carlo generator parameters,
%-----------------------------------------------------------------------
$$
R_{u/d} =\frac{
f(c\rightarrow D^{0})
-f(c\rightarrow D^{*+}){\cal B}_{D^{*+}\rightarrow D^0} 
}
{
f(c\rightarrow D^{+})
+f(c\rightarrow D^{*+}){\cal B}_{D^{*+}\rightarrow D^0} 
},
$$
$$
\gamma_{s}=\frac{2f(c\rightarrow D^+_{s})}{f(c\rightarrow D^+)+
f(c\rightarrow D^0)}
$$ 
and 
%-----------------------------------------------------------------------
$$
P^d_{V} =\frac{f(c\rightarrow D^{*+})+f(c\rightarrow D^{*0})}
{f(c\rightarrow D^{+})+f(c\rightarrow D^{0})}
$$
%-----------------------------------------------------------------------
were calculated from the fit results with the full error propagation and
taking into account the correlation between parameters.
\FloatBarrier
%%%%%%%%%%%%%%%%%%%%%%%%%%%%%%%%%%%%%%%%%%%%%%%%%%%%%%%%%%%%%%%%%%%%%%%%
\section{Charm-quark fragmentation into hadrons in {\pmb \epem} 
collisions}
Measurements of charm-hadron-production cross-sections in \epem 
collisions hadrons were based on the differential momentum spectrum 
$\mathrm{d}\sigma(e^+e^- \rightarrow H)/\mathrm{d}x_p.$
%
The extrapolation to
the total cross-section was made in the original papers
using a theoretical
fragmentation function (e.g.\ Bowler~\cite{Bowler:1981sb} or 
Peterson~\cite{Peterson:1982ak}) 
\footnote{A proper extrapolation procedure requires 
only the  hadrons produced directly in fragmentation to be used in the 
fits. 
%
The hadrons produced in decays of excited charm hadrons should be 
treated separately.
 
%
In many cases the limited precision of the measurements 
makes this requirement hard to follow and the decay part of the meson 
production is treated together with the fragmentation part. 
%
In the cases 
where the contribution of hadrons from decays is comparable with the 
contribution of the direct production in fragmentation, e.g.\ for $D^0$ 
and $D^+$, the joint treatment could bias the results.
}. 

As mentioned before, the precise predictions of the total 
charm-production cross-section in \epem allow calculation of
 the fragmentation
 fractions without constraints on the sum of fractions.
%
This way the used hypothesis about the sum of fragmentation fractions 
(\eq{ffsumisone}) can be verified.
%%%%%%%%%%%%%%%%%%%%%%%%%%%%%%%%%%%%%%%%%%%%%%%%%%%%%%%%%%%%%%%%%%%%%%%%
\subsection{Charm-quark fragmentation fractions from measurements at 
${\pmb B}$-factories}
\label{sec:ssUPSILON}
The $B$-factories provided many results on   charm-hadron production 
around the $\Upsilon$ resonances, which can be used for the calculation 
of the charm-quark fragmentation fractions in hadrons (see 
\tab{EEUmeas}). 
The results  of the CLEO~\cite{Bortoletto:1988kw,Avery:1990bc}
 and ARGUS~\cite{Albrecht:1991ss,
Albrecht:1991pa,
Albrecht:1988an} 
experiments are represented as a product of the
charm-hadron, $H$, cross-sections times decay branching ratios, 
$\sigma(e^+e^-\rightarrow H)\cdot {\cal B}(H\rightarrow 
\text{non--charm hadrons}).$
%
The BELLE experiment~\cite{Seuster:2005tr} provided measurements of 
$\sigma(e^+e^-\rightarrow  H)$. 
%
The BABAR experiment~\cite{Aubert:2006cp} provided a measurement of an 
average number of $\Lambda_c^+\rightarrow p K\pi$ decays per hadronic 
event 
\begin{multline*}
N^{q\bar{q}}_{\Lambda_c} \cdot {\cal B}(\Lambda_c^+\rightarrow p K\pi)=
2\frac{\sigma(e^+e^-\rightarrow  \Lambda_c^+)}{\sigma(e^+e^-\rightarrow 
\text{hadrons})} \cdot {\cal B}(\Lambda_c^+\rightarrow p K\pi)=\\
=2R_c \cdot f(c\rightarrow\Lambda_c^+) \cdot {\cal B}(\Lambda_c^+
\rightarrow p K\pi),
\end{multline*}
where 
$R_{c}=\frac{\sigma(e^+e^-\rightarrow c\bar{c})}{\sigma(e^+e^-
\rightarrow \text{hadrons})}$ is the average number of charm-quark pairs
 per hadronic event.
%
A prediction of $R_c$ is needed to use this measurement as an input, 
therefore, as discussed in \sect{comb:proc}, it is  used only in the 
case of $\sigma(e^+e^- \rightarrow c\bar{c})$ fixed to a theoretical 
prediction.
\tabEEUmeas

%
For the calculation of the charm-quark fragmentation fractions a fit 
procedure is used as  described in \sect{comb}. The total 
charm-quark-production  cross-section is calculated as described in
 \app{appendixA}. 
%
%
The fit parameters 
are the fragmentation fractions. The obtained results are given  in the 
middle column of
\tab{EEUaverage}.
%
The sum of the charm-quark  fragmentation fractions into weakly decaying 
states calculated according to \eq{ffsumisone}, 
$S_{\Upsilon}=\SUpsilon,$ is consistent with unity.
%
\tabEEUaverage
%

The combination is also done according to \eq{ffwithsumh} and  imposing 
the constraint $S_{\Upsilon}-1=0$, to be consistent with the definition 
used for $e^{\pm}p$ and $pp$ data. The fit parameters are the 
fragmentation fractions and the total charm cross-section. 
The centre-of-mass energy dependence of the charm-quark cross-section 
is accounted for, according to formulae in \app{appendixA} taking the 
total charm-quark 
cross-section  at a centre-of-mass energy
 $\sqrt{s}=\sqrtS~\GeV$  as a reference.
 The results are given  in \tab{EEUaverage}~(right column).
In this approach, the precise BABAR measurement of $\Lambda_{c}^{+}$ 
production~\cite{Aubert:2006cp} is not included in the combination
 since it requires usage of the $R_c$ theoretical prediction.
The latter has an  influence on other fragmentation-fraction results.
\FloatBarrier
%%%%%%%%%%%%%%%%%%%%%%%%%%%%%%%%%%%%%%%%%%%%%%%%%%%%%%%%%%%%%%%%%%%%%%%%
\subsection{Charm-quark fragmentation fractions from measurements at 
LEP}
\label{sec:ssZZERO}
The LEP collider provided many results on the  charm-hadron production. 
%
The most valuable for the studies of fragmentation are results obtained 
from $Z$ decays. 
%
Most of those results are represented in the form of fraction of charm 
events  multiplied by branching ratios
$\frac{\Gamma(Z\rightarrow c\bar{c})}
{\Gamma(Z\rightarrow \text{hadrons})}$
$f(c\rightarrow H)\cdot {\cal B}(H\rightarrow\text{non--charm~hadrons})$
(see \tab{EEZmeas}).
%
In addition,  ALEPH~\cite{Barate:1999bg}, DELPHI~\cite{Abreu:1999vx} and 
OPAL~\cite{Ackerstaff:1997ki} provided measurements of $f(c \rightarrow 
D^{*+})$ from 
\tabEEZmeas
the fits of fragmentation functions (see \tab{EEZmeas}).  
%

For the calculation of charm-quark fragmentation fractions, a fit 
procedure is used, as  described in \sect{comb}. 
The theoretically 
calculated value that is used, $\frac{\Gamma(Z \rightarrow 
c\bar{c})}{\Gamma(Z \rightarrow \text{hadrons})}
=0.17223\pm0.00001$~\cite{Freitas:2014hra}, is in agreement with
the experimental world average 
$0.1721\pm0.003$~\cite{PDG2014}.
%
The fit parameters are the fragmentation fractions.
%
The results are given in the middle column of \tab{EEZaverage}.
The sum of the charm-quark  fragmentation fractions into  weakly 
decaying 
states calculated according to \eq{ffsumisone}, $S_{Z}=\SZ,$  
differs from unity by $\SZsign$ standard deviations.
\FloatBarrier
\tabEEZaverage
%

The combination is also done using \eq{ffwithsumh}, and imposing 
the constraint $S_{Z}-1=0$, to be consistent with the definition 
used for $e^{\pm}p$ and $pp$ data. The fit parameters are the 
fragmentation 
fractions and the $\frac{\Gamma(Z \rightarrow c\bar{c})}{\Gamma(Z 
\rightarrow \text{hadrons})}$ ratio. The results, given in 
\tab{EEZaverage} (right column), 
are in good agreement with Ref.~\cite{Gladilin:2014tba}.
\FloatBarrier
%%%%%%%%%%%%%%%%%%%%%%%%%%%%%%%%%%%%%%%%%%%%%%%%%%%%%%%%%%%%%%%%%%%%%%%%
\section{Charm-quark fragmentation into hadrons in {\pmb\ep} collisions}
\label{sec:ffep}
The charm-hadron-production cross-sections at HERA were measured in a 
restricted fiducial phase space. 
%
The extraction of the charm-quark fragmentation fractions  requires a 
special treatment, as described 
in detail in \app{appendixB}.
The approach followed in this analysis is  similar to the one originally 
used by the ZEUS collaboration~\cite{Chekanov:2005mm}.
%%%%%%%%%%%%%%%%%%%%%%%%%%%%%%%%%%%%%%%%%%%%%%%%%%%%%%%%%%%%%%%%%%%%%%%%
\FloatBarrier
\subsection{Charm-quark fragmentation fractions from measurements in 
DIS}
Charm-quark fragmentation fractions in DIS in $e^{\pm}p$ collisions are 
calculated from ZEUS and H1 measurements given in \tab{DISmeas}.
\FloatBarrier
\tabDISmeas
For the calculation of charm-quark fragmentation fractions a fit 
procedure is used as it is described in \sect{comb}.
The free parameters in the fit are the charm fragmentation fractions
and  pairs  of variables $\sigma(c)_{i}|_{i=1\dots 3}$ and 
$\kappa_{i}|_{i=1\dots 3}$ for each set of measurement. 
%
Here, $\sigma(c)_{i}$ is the total charm cross-section in \ep, while 
$\kappa_{i}$ is the kinematic factor for decays from higher states 
(see \app{appendixB}). 
%
The parameter $\kappa$ is fixed to one for the low-$p_T$ measurements 
in Ref.~\cite{Abramowicz:2010aa} since the whole $p_T$ kinematic space 
was covered.
The sum of charm fragmentation fractions $S_{ep~\text{DIS}}$ is 
constrained to unity.
The results of the averaging procedure are given in \tab{DISaverage}.
\tabDISaverage

The obtained fragmentation fractions are in agreement with those 
obtained in the original publications~\cite{Chekanov:2007ch,
Aktas:2004ka}. 
%
The uncertainties of the obtained results  are somewhat larger  because 
this analysis, contrary to those studies, relies only on DIS results, 
whereas the HERA DIS papers~\cite{Chekanov:2007ch,
Aktas:2004ka} used fragmentation fractions into 
$\Lambda_{c}^{+}$ measured at \epem colliders.
\FloatBarrier
%%%%%%%%%%%%%%%%%%%%%%%%%%%%%%%%%%%%%%%%%%%%%%%%%%%%%%%%%%%%%%%%%%%%%%%%
\subsection{Charm-quark fragmentation fractions from measurements in 
PHP}
Charm-quark fragmentation fractions in PHP in $e^{\pm}p$ collisions
 were calculated from measurements of  ZEUS and H1
            collaborations and given in \tab{PHPmeas}.
\FloatBarrier
\tabPHPmeas
%
For the update of the latest ZEUS measurement~\cite{Abramowicz:2013eja} 
to the  decay branching ratios from \tab{BIGPDG}, the measured 
fragmentation fractions are first transformed into total charm-hadron 
cross-sections according to the formulae in \app{appendixB} and only 
then used in the calculations. In this procedure, the kinematic factor 
for decays from higher states, $\kappa$, is set to $1$, since  the total 
phase space is considered from the fragmentation fraction definition, 
and the $\sigma(c)$ value cancels out in the procedure.
%

%
For the calculation of charm-quark fragmentation fractions, a fit 
procedure is used as it is described in \sect{comb}.
The free parameters in the fit are the charm fragmentation fractions
and  pairs  of variables $\sigma(c)_{i}|_{i=1 \dots 2}$ and 
$\kappa_{i}|_{i=1 \dots 2}$ for each set of measurement. 
%
Here, $\sigma(c)_{i}$ is the total charm cross-section in \ep, while 
$\kappa_{i}$ is the kinematic factor for decays from higher states 
(see \app{appendixB}). 
%
The sum of charm fragmentation fractions $S_{ep~\text{PHP}}$ is 
constrained to unity.
%
The results of the averaging procedure are given in \tab{PHPaverage}.
%
The obtained fragmentation fractions are in agreement with those 
obtained in the original publications~\cite{Chekanov:2005mm,
Abramowicz:2013eja}.
%
\FloatBarrier
\tabPHPaverage
%

%%%%%%%%%%%%%%%%%%%%%%%%%%%%%%%%%%%%%%%%%%%%%%%%%%%%%%%%%%%%%%%%%%%%%%%%
\FloatBarrier
\section{Charm-quark fragmentation into hadrons in 
${\pmb {pp}}$ collisions}
\label{sec:ffpp}

The LHCb experiment recently provided  measurements of charm-baryon 
cross-sections at $\sqrt{s}=7~\TeV$~\cite{Aaij:2013mga}. 
%
The measurements together with the correlation matrix are given in 
\tab{LHCBmeas}.
%
\FloatBarrier
\tabLHCBmeas
%

For the calculation of charm-quark fragmentation fractions a fit 
procedure is used as it is described in \sect{comb}. The free 
parameters in the fit are the charm fragmentation fractions and the 
fiducial cross-section $\sigma(pp\rightarrow c)$.
%
The constraint $S_{pp}-1=0$ is imposed.
%
A set of orthogonal fully correlated uncertainties was obtained from the 
 covariance matrix of the LHCb measurements with a eigenvector 
 decomposition. The obtained uncertainties  are further treated in the 
 same way as other correlated sources in the combination.
%
The results of the fit are reported in \tab{LHCBaverage}.
In this fit, the number of fitted parameters is equal to the number 
of measured quantities resulting  $n_{\text{dof}}=0$.
\FloatBarrier
\tabLHCBaverage
%%%%%%%%%%%%%%%%%%%%%%%%%%%%%%%%%%%%%%%%%%%%%%%%%%%%%%%%%%%%%%%%%%%%%%%%
\FloatBarrier
\section{Selection of measurements for the extraction of fragmentation 
fractions}

The selection of the measurements for the extraction of fragmentation 
fractions was done according a set of criteria explained below.

First, the selection is limited to the measurements obtained in the 
collisions of particle beams as it assures an absence of possible matter 
effects: the measurements of charm-hadron production in 
proton--meson, proton--nucleon and nucleon--nucleon 
collisions~\cite{Alves:1996rz,Barlag:1990bv,
Barlag:1990hg,Abt:2007zg,Tlusty:2012ix,Ye:2014eia} were omitted as 
those provide results in very specific production environment and energy 
ranges  which cannot be easily compared to the results in other 
experiments.

The second criteria of the selection is the  precision of the measured 
quantities:
 the measurements in \epem collisions with $\sqrt{s}=12-90~\GeV$
 from 
 MARK-II~\cite{Yelton:1982ix}, HRS~\cite{Ahlen:1983gz,Derrick:1984ba,
 Derrick:1985ip,Low:1986nz,Baringer:1988ue}, TPC~\cite{Aihara:1985pp},
TASSO~\cite{Althoff:1983rt,Braunschweig:1989kq}, JADE~\cite{
Bartel:1984ud,Bartel:1985be}, VENUS~\cite{Hinode:1993gj} and some other 
experiments  have very limited precision and are not used for the 
global 
combination. 

The third criterion of the selection is the availability of sufficient 
measurements in the given physical environment needed for the extraction 
procedure. 
Several results  on charm production in $e^{\pm}p$ collisions (e.g.\ 
Ref.~\cite{Chekanov:2008yd}) and $pp$ collisions (e.g.\  Refs.
~\cite{Tlusty:2012ix,Ye:2014eia,Acosta:2003ax}) do not contain enough
 simultaneous 
measurements of hadron production and, therefore, cannot be treated 
independently
and/or constrain the results of the combination.
% 
The fourth criterion of the selection is a minimal model dependence of
the results:  the results from ALICE~\cite{Abelev:2012tca,
Abelev:2012vra,ALICE:2011aa} contain significant theory related 
uncertainties, which exceed the experimental precision of these 
measurements. 
%
Preliminary results from ATLAS~\cite{ATLAS:2011fea} are also not 
included in the present combination.
%%%%%%%%%%%%%%%%%%%%%%%%%%%%%%%%%%%%%%%%%%%%%%%%%%%%%%%%%%%%%%%%%%%%%%%%
\FloatBarrier
\section{The global combination} 
\label{sec:ffglob}
To check the consistency of the data from different production regimes 
and also to extract the charm-quark fragmentation fractions with  
high precision, all input measurements introduced in the previous 
sections
 are used together to produce a global combination.
%
As discussed in \sect{ssUPSILON}, the $\Lambda_{c}^{+}$ measurement by 
the BABAR  experiment~\cite{Aubert:2006cp} is not included while 
obtaining the combined result. 
%
The free parameters of the fit are the charm-quark fragmentation 
fractions and  pairs of variables $\sigma(c)_{i}|_{i=1\dots 5}$ and 
$\kappa_{i}|_{i=1\dots 5}$  for three DIS and two PHP sets of 
measurements, $\frac{\Gamma_{c\bar c}}{\Gamma_{\text{had}}}$, 
$\sigma(e^+e^-\rightarrow c)$ at 
$\sqrt{s}=\sqrtS~\GeV$, 
and the fiducial charm-quark cross-section in $pp$, 
$\sigma(pp\rightarrow c)$, at $\sqrt{s}=7~\TeV$, 
corresponding to the phase space of the measurement. 
%
The constraint on the sum of the cross-sections of the weakly decaying
charm states, $S$, is imposed in the combination, i.e.\ the  
prediction for the total charm cross-sections in \epem collisions
is not used, in order to minimise model dependence
of the averaging procedure. 
The result of averaging \epem, $e^{\pm}p$ and $pp$ data, with the 
constraint $S=1$ is presented in
the middle column of
\tab{FINALaverage} and is shown in \fig{FINALaverageFF}. The 
correlations between the fitted parameters are given in 
\tab{FINALcorrelations}.
%
The input data are in very good agreement with  $\chi^2 / n_\text{dof} 
=\fitgoodness.$
%
The result of the combination has significantly reduced uncertainties 
compared to individual measurements.
%

\figFINALaverageFF
\tabFINALaverage
\tabFINALcorrelations

As an alternative, the combination ia also performed using both the 
constraint on $S$ as well as theoretical predictions of charm 
production
in \epem collisions and $Z$ decays, i.e.\  $\sigma(e^+e^-\rightarrow c)$
at $\sqrt{s}=\sqrtS~\GeV$
and $\frac{\Gamma_{c\bar c}}{\Gamma_{\text{had}}}$. 
%
This approach also allows to include the precise BABAR 
measurement of $\Lambda_{c}^{+}$ production~\cite{Aubert:2006cp} to 
be included using the $R_c$ calculation as described in \app{appendixA}, 
which significantly affects the averaged value of $f(\Lambda_c^+)$.
%
The result of the averaging procedure with this approach
 is given in the right column of \tab{FINALaverage} for completeness.
The result is more model dependent than the default 
combination, but has a higher precision. At the same time, the result 
visibly differs from the result of the default procedure.
%
This may partially be
traced to the value $S_{Z} = \SZ$ for the accurate LEP
measurements, which differs markedly from $1$ (see \sect{ssZZERO}).
%
This difference is also reflected in the larger $\chi^2 /n_{\text{dof}}$
value compared to the default global combination.
%
The difference in the $f(c \rightarrow \Lambda_{c}^{+})$ precision is to
a large extent due to inclusion of the precise BABAR 
 data~\cite{Aubert:2006cp}.

\figFINALaverageRPg
The $R_{u/d}$, $P^d_{V}$ and $\gamma_{s}$ factors are also extracted 
and are provided in \tab{FINALaverage} and compared in 
\fig{FINALaverageRPg}.
%
The combined data are also compared to recent 
measurements~\cite{ATLAS:2011fea,ALICE:2011aa,
Abelev:2012vra, Abelev:2012tca, Acosta:2003ax, David:2007iv}
that were not included in the combination.
%
In particular, $R_{u/d} = \RudG\,$ is in fair agreement with the isospin 
invariance hypothesis $R_{u/d} = 1$ within $\RudGsign$ standard 
deviations.
%
\FloatBarrier
%%%%%%%%%%%%%%%%%%%%%%%%%%%%%%%%%%%%%%%%%%%%%%%%%%%%%%%%%%%%%%%%%%%%%%%%
\section{Excited states}  
In addition to the average fragmentation fractions for the ground, 
$L=0$, states, some fragmentation fractions for the excited, 
$L=1$ charm hadrons are calculated.

The  measurements used for the calculations are shown in  \tab{excited}.
The unpublished measurement of $f(c\rightarrow D^+_{s1})$ from 
Ref.~\cite{Verbytskyi:2013jsa} was not used.
The fragmentation fractions were not updated to the most recent 
branching ratios, as the difference between the used branching ratios 
and the newest is  negligible in comparison to statistical and 
systematical  uncertainties of the measurements, and is well below the 
given numerical precision of the individual measurements in 
\tab{excited}.

\tabexcited
The averages are calculated with an assumption of fully uncorrelated 
statistical and systematical uncertainties. The results of the averaging 
procedure as well as  the strangeness-suppression  factor 
for $1^{+}$ charm mesons, 
$$
\gamma_{s1}=\frac{2f(c\rightarrow D^+_{s1})}
{f(c\rightarrow D^0_1)+f(c\rightarrow D^+_1)},
$$ are given in \tab{FINALexcited}.
%The \tab{FINALexcited} also contains 
\FloatBarrier
\tabFINALexcited
\FloatBarrier
%%%%%%%%%%%%%%%%%%%%%%%%%%%%%%%%%%%%%%%%%%%%%%%%%%%%%%%%%%%%%%%%%%%%%%%%
\section{Summary}                      
\label{sec:con}
A summary of measurements of the fragmentation of charm quarks into a
specific charm hadron is given. 
%
The analysis includes data collected in photoproduction and deep 
inelastic scattering in $e^{\pm}p$ collisions and well as \epem and $pp$
 data.
%
Measurements in different production regimes agree within uncertainties, 
supporting the hypothesis that fragmentation proceeds independent of the 
specific production process. 
%
Averages of the fragmentation fractions are presented. 
%
The global average has significantly reduced uncertainties compared 
to individual measurements.
%
In addition, the hypothesis that the sum of fragmentation fractions 
of all known weakly decaying charm hadrons is equal to unity is
checked to hold within $3$ standard deviations 
using the \epem data.
%
%%%%%%%%%%%%%%%%%%%%%%%%%%%%%%%%%%%%%%%%%%%%%%%%%%%%%%%%%%%%%%%%%%%%%%%%
\section*{Acknowledgements}
\label{sec:ack}
We thank Erich Lohrmann for his major contribution to the development 
of this paper. We thank  Uri Karshon and Stefan Kluth  for useful 
discussions and help in the work with the bibliography.
We also thank  Alexander  Glazov  and Ian Brock
for the critical reading of the manuscript and 
useful suggestions on text improvement.
\FloatBarrier
%\newpage
\begin{appendices}
%%%%%%%%%%%%%%%%%%%%%%%%%%%%%%%%%%%%%%%%%%%%%%%%%%%%%%%%%%%%%%%%%%%%%%%%
\section{Predictions for charm production at ${\pmb B}$-factories}
\label{sec:appendixA}
The  total   cross-section of quark $q$ production in \epem collisions 
at energies $2M(q) \ll \sqrt{s} \ll M(Z)$ can be given as
\begin{equation}\label{eq:cseeqq}
\sigma(e^+e^-\rightarrow q/\bar{q})=
2(\Sigma_{\text{colours}} v_q^2  r_q)\sigma(e^+e^-\rightarrow l^+l^-),
\end{equation}
where $v_q$ is the vector electromagnetic coupling of the quark $q$ 
(i.e.\  charge), $r_q(s)$ are the correction coefficients with higher 
order QCD corrections and
\begin{equation}\label{eq:cs}
\sigma(e^+e^-\rightarrow l^+l^-)=\frac{4\alpha^2(s)\pi}{3s}
\end{equation}
is the  total  cross-section of massless charged lepton pair production.
In this work, the calculations of the $r_q(s)$ were done according to 
Ref.~\cite{Chetyrkin:2000zk} at the reference energy of 
$\sqrt{s}=\sqrtS$  and assuming the $c$ quark is the heavy one. The 
constants used for the calculations in \eq{cseeqq} and \eq{cs} are
the strong coupling $\alpha_s(\sqrt{s}=\sqrtS~\GeV)=0.172$
~\cite{Chetyrkin:2000zk}, the $\overline{\text{MS}}$ charm-quark mass 
$m_c(\sqrt{s}=\sqrtS~\GeV)=0.74~\GeV$~\cite{Chetyrkin:2000zk} and 
the electromagnetic coupling $\alpha(\sqrt{s}=\sqrtS~\GeV)=1/132.0$ 
(calculated according to~\cite{Altarelli:1989hx,Burkhardt:1989ky} as 
implemented in~\cite{Harris:1997zq}). The uncertainties on the given 
values are negligible. The result of the calculations is 
$\sigma(e^+e^-\rightarrow c/\bar{c},\sqrt{s}=\sqrtS~\GeV)=\sigmaeecc\,
\text{nb}$.
To verify the calculations,  
\eq{cseeqq}  can be rewritten as 
\begin{equation*}\label{eq:sigmactwo}
\sigma(e^+e^-\rightarrow c)=2 R_c R_{\text{hadrons}} 
\sigma(e^+e^-\rightarrow l^{+}l^{-}),
\end{equation*} where the quantities
\begin{equation*}\label{eq:rh}
R_{\text{hadrons}}=
\frac{\sigma(e^+e^-\rightarrow \text{hadrons})}{\sigma(e^+e^-\rightarrow 
l^+l^-)}=\Sigma_{\text{quarks=}u,d,s,c}\Sigma_{\text{colours}} v_q^2 
r_q=\Rhadrons
\end{equation*}
and 
\begin{equation*}\label{eq:rc}
R_{c}=\frac{\sigma(e^+e^-\rightarrow c\bar{c})}{\sigma(e^+e^-\rightarrow 
\text{hadrons})}=\frac{\Sigma_{\text{colours}} v_c^2 r_c}
{\Sigma_{\text{quarks=}u,d,s,c}\Sigma_{\text{colours}} v_q^2 r_q}=\Rc
\end{equation*}
can be compared with the existing measurements and predictions.
It was found that $R_{\text{hadrons}}$ is in agreement with the direct  
measurement from CLEO below $\sqrt{s}=10.56~\GeV$
$R_{\text{hadrons,CLEO}}=3.591\pm0.003\pm0.067\pm0.049$
~\cite{Besson:2007aa} 
and $R_{c}$ is in  agreement with the CLEO Monte-Carlo based estimation
$R_{\text{c,CLEO}}=0.37\pm0.05$~\cite{Bortoletto:1988kw}.
For all the theoretically calculated values, the 
uncertainties of calculations are negligible.
\FloatBarrier
%%%%%%%%%%%%%%%%%%%%%%%%%%%%%%%%%%%%%%%%%%%%%%%%%%%%%%%%%%%%%%%%%%%%%%%%
\section{Extraction of the charm-quark fragmentation fractions at HERA}
\label{sec:appendixB}
The measurements of fragmentation fractions in $e^{\pm}p$ collisions are
 based on  the measurements of cross-sections of charm hadrons in a 
restricted (e.g.\ in transverse momentum and pseudorapidity) kinematical 
 region $v$. In a general case it implies
$$f(c \rightarrow   H,v)= 
\frac{\sigma(ep\rightarrow H, v)}{\sigma(ep\rightarrow c, v)}\neq f(c 
\rightarrow   H).$$
As the masses and quark content of different charm hadrons is different, 
their momentum-dependent  fragmentation functions are not exactly the 
same
$$\sigma(ep\rightarrow c, v) \neq \Sigma_{i}\sigma(ep\rightarrow H_{i}, 
v).$$
However, for the calculation the difference was checked to be 
small~\cite{Chekanov:2005mm} and therefore was neglected\footnote{
Partially, the difference can be covered by the systematic uncertainty
 evaluated as a variation of the fragmentation functions performed in 
 the measurements.}.
The full   cross-section of a  charm hadron $H$, $\sigma(ep\rightarrow 
H, v)_{\text{vis}}$ can be given as a sum of cross-sections of direct 
hadronisation $\sigma(ep\rightarrow  D, v)_{\text{dir}}$ and the 
contribution of decays of other states  
$\sigma(ep\rightarrow  D, v)_{\text{decays}}$:
\begin{multline*}
\sigma(ep\rightarrow H, v)_{\text{vis}}= 
\sigma(ep\rightarrow  H, v)_{\text{dir}}+
\sigma(ep\rightarrow  H, v)_{\text{decays}}=\\
= \sigma(ep\rightarrow  H, v)_{\text{dir}}+
\Sigma_{i} \sigma(ep\rightarrow  H^{*}_i)_{\text{dir}}    
B_{H^{*}_{i} \rightarrow H}k^{*}_i(v),
\end{multline*}
where $B_{H^{*}_{i} \rightarrow H}$ are the decay branching ratios of 
hadrons $H^{*}_{i}$ to $H$ and $k^{*}_i(v)<1$ is a fraction of $H^{*}_i 
\rightarrow H$ decays with $H$  in $v$. With an assumption that for the 
higher states $\sigma(c\rightarrow H^*, v)_{\text{dir}}=
\sigma(c\rightarrow H^*, v)_{\text{vis}}$ and  equal fragmentation 
functions for all hadrons $k^{*}(v)_j=k^{*}(v)_i=\kappa$ we have
$$
\sigma(ep\rightarrow H, v)_{\text{vis}} = 
\sigma(ep\rightarrow  H, v)_{\text{dir}}+   
\Sigma_{i}      \kappa       \sigma(ep\rightarrow  H^{*}_i)_{\text{dir}}
  B_{H^{*}_{i} \rightarrow H}.
$$

As all excited single charm non-strange states decay exclusively to 
$\Lambda_c^+$ and all excited charm-strange meson  states decay 
exclusively to $D^{+}_s$ we can treat all 
$\Lambda_c^{+}$ and  $D^{+}_s$ as produced directly. Hereby, we have the
equations: 
$$
\begin{cases}
\sigma(ep\rightarrow D^+, v)_{\text{vis}}  =
\sigma(ep\rightarrow  D^+, v)_{\text{dir}}+\kappa \sigma(ep\rightarrow  
D^{*+})_{\text{dir}}B_{D^{*+} \rightarrow D^+}\\
\sigma(ep\rightarrow D^0, v)_{\text{vis}}  =\sigma(ep\rightarrow  D^0, 
v)_{\text{dir}}
+\kappa \sigma(ep\rightarrow  D^{*0})_{\text{dir}}B_{D^{*0} \rightarrow 
D^0}+\\
\hfill                                                                  
\kappa\sigma(ep\rightarrow  D^{*+})_{\text{dir}}B_{D^{*+}\rightarrow 
D^0}\\
\sigma(ep\rightarrow D^+_s,v)_{\text{vis}}=
\sigma(ep\rightarrow  D^+_s,v)_{\text{dir}}\\
\sigma(ep\rightarrow \Lambda, v)_{\text{vis}}=
\sigma(ep\rightarrow  \Lambda, v)_{\text{dir}}\\
\sigma(ep\rightarrow D^{*+}, v)_{\text{vis}}=
\sigma(ep\rightarrow  D^{*+}, v)_{\text{dir}}\\
\sigma(ep\rightarrow D^{*0}, v)_{\text{vis}}=
\sigma(ep\rightarrow  D^{*0}, v)_{\text{dir}}
\end{cases}
$$

Assuming  $f(c \rightarrow H, v)_{\text{dir}}=f(c 
\rightarrow H)_{\text{dir}}$ we have the following equations:
$$
\begin{cases}
\sigma(ep\rightarrow D^+, v)_{\text{vis}}      
 =\sigma(ep\rightarrow c, v)
(f(c \rightarrow  D^+)_{\text{dir}}+ 
\kappa f(c \rightarrow   D^{*+})_{\text{dir}}
B_{D^{*+} \rightarrow D^+})\\
\sigma(ep\rightarrow D^0, v)_{\text{vis}}       
=\sigma(ep\rightarrow c, v)
(f(c \rightarrow   D^0)_{\text{dir}}+
\kappa f(c \rightarrow   D^{*0})_{\text{dir}}
B_{D^{*0} \rightarrow D^0}+\\
\hfill\kappa f(c \rightarrow  
D^{*+})_{\text{dir}}B_{D^{*+} \rightarrow D^0})\\
\sigma(ep\rightarrow D^+_s, v)_{\text{vis}}     
=\sigma(ep\rightarrow c, v)
f(c \rightarrow   D^+_s)_{\text{dir}}\\
\sigma(ep\rightarrow \Lambda, v)_{\text{vis}}   
=\sigma(ep\rightarrow c, v)
f(c \rightarrow   \Lambda)_{\text{dir}}\\
\sigma(ep\rightarrow D^{*+}, v)_{\text{vis}}      
 =\sigma(ep\rightarrow c, v)
f(c \rightarrow   D^{*+})_{\text{dir}}\\
\sigma(ep\rightarrow D^{*0}, v)_{\text{vis}}       
=\sigma(ep\rightarrow c, v)
f(c \rightarrow   D^{*0})_{\text{dir}}.
\end{cases}
$$
In the full kinematical space:
$$
\begin{cases}
f(c \rightarrow  D^+)_{\text{vis}}       
= f(c \rightarrow  D^+)_{\text{dir}}          
+ f(c \rightarrow   D^{*+})_{\text{dir}}B_{D^{*+} \rightarrow D^+}\\
f(c \rightarrow  D^0)_{\text{vis}}       
= f(c \rightarrow   D^0)_{\text{dir}}          
+ f(c \rightarrow   D^{*0})_{\text{dir}} B_{D^{*0} \rightarrow D^0}+ 
f(c \rightarrow   D^{*+})_{\text{dir}}B_{D^{*+} \rightarrow D^0}\\
f(c \rightarrow  D^+_s)_{\text{vis}}     
=f(c \rightarrow   D^+_s)_{\text{dir}}\\
f(c \rightarrow  \Lambda)_{\text{vis}}   
=f(c \rightarrow   \Lambda)_{\text{dir}}\\
f(c \rightarrow  D^{*+})_{\text{vis}}       
=f(c \rightarrow   D^{*+})_{\text{dir}}\\
f(c \rightarrow  D^{*0})_{\text{vis}}       
=f(c \rightarrow   D^{*0})_{\text{dir}}.
\end{cases}
$$
In general, to solve the system the measurements of $D^{*0}$ production 
are needed. However, these can be avoided with an assumption of isospin 
invariance:$$\frac{f(c\rightarrow D^+)_{\text{dir}}}
{f(c\rightarrow D^0)_{\text{dir}}}
=\frac{f(c\rightarrow D^{*+})_{\text{dir}}}{f(c\rightarrow 
D^{*0})_{\text{dir}}}.$$
The last two systems are the working equations for the calculation of 
the  charm fragmentation fractions from the cross-section measurements 
in $ep$ collisions.
\end{appendices}
