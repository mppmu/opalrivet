\documentclass[zpreprint,zbstnp,final]{zeus_paper}
%\usepackage[english]{babel}
\usepackage{placeins}
\usepackage[titletoc,title]{appendix}
\usepackage{authblk}
\usepackage{lineno}
\usepackage{hyperref}
\linenumbers
\title{Combined analysis of charm-quark fragmentation-fraction measurements}
\author[a]{Mykhailo Lisovyi\thanks{mikhaylo.lisovyi@desy.de}}
\author[b]{Andrii Verbytskyi\thanks{andrii.verbytskyi@mpp.mpg.de}}
\author[c]{Oleksandr Zenaiev\thanks{oleksandr.zenaiev@desy.de}}

\affil[a]{Physikalisches Institut der Universit\"{a}t Heidelberg
}
\affil[b]{Max-Planck-Institut f\"{u}r Physik
}
\affil[c]{on leave from DESY
}

\renewcommand\Authands{\\and\\}

%%%%%%%%%%%%%%%%%%%%%%%%%%%%%%%%%%%%%%%%%%%%%%%%%%%%%%%%%%%%%%%%%%%%%%%%
%example
%%%%%%%%%%%%%%%%%%%%%%%%%%%%%%%%%%%%%%%%%%%%%%%%%%%%%%%%%%%%%%%%%%%%%%%%

\newcommand{\TABexample}{
\begin{table}[htbp]\centering
\begin{tabular}{|c|c|c|}\hline
Mode      &  Used in this work (\%) & Used in experiments (\%)\\\hline\hline
%\input{Tables/pdg_tab.txt}
\end{tabular}
\caption{
Example
Ref.~\protect\cite{Zupanc:2013iki}. 
}
\label{tab:example}
\end{table}
}


\newcommand{\TABluminocity}{
\begin{table}[htbp]\centering
\begin{tabular}{|c|c|c|c|}\hline
Year      &  $\sqrt{s}$ range,\GeV & $\sqrt{s}$,\GeV & Luminocity,$pb^{-1}$\\\hline\hline
 1995,1997 & $129.9-130.2$ & $130.1$ & $2.7$ \\
 1995,1997, & $129.9-130.2$ & $130.1$ & $2.61$  \\
 1995,1997 & $136.0-136.3$ & $136.1$ & $2.56$  \\
 1995,1997 & $136.0-136.3$ & $136.1$ & $2.36$  \\
 1996 & $161.2-161.6$ & $161.3$ & $10.06$  \\
 1996 & $170.2-172.5$ & $172.1$ & $10.06$  \\
 1997 & $180.8-184.2$ & $182.7$ & $57.72$  \\
 1999 & $188.3-189.1$ & $189.1$ & $185.2$  \\
 1999 & $191.4-192.6$ & $192.1$ & $29.53$  \\
 1999,2000 & $195.4-196.1$ & $195.5$ & $76.67$  \\
 1999,2000 & $201.3-202.1$ & $199.5$ & $37.97$  \\
 2000 & $202.5-205.5$ & $204.9$ & $81.96$  \\
 2000 & $205.5-208.9$ & $206.6$ & $137.14$  \\
%\begin{tabular}{ l r@{---}l l l r }
%\hline\noalign{\smallskip}
%ear       & \multicolumn{2}{l}{Range of \rs   }  & Mean $\rs$   & ${\cal L}$    & Selected \\
 %          & \multicolumn{2}{l}{[GeV]}  & [GeV]        & [pb$^{-1}$]   & events   \\
%\noalign{\smallskip}\hline\noalign{\smallskip}
%1996, 2000 &  91.0 & 91.5 &  91.3 &  14.7 & 395695 \\ 
%\hline
%1995, 1997 & 129.9 & 130.2 & 130.1 &   5.31 & 318   \\ 
%1995, 1997 & 136.0 & 136.3 & 136.1 &   5.95 & 312   \\ 
%\hline
%1996       & 161.2 & 161.6 & 161.3 & 10.06 & 281    \\ 
%1996       & 170.2 & 172.5 & 172.1 & 10.38 & 218    \\ 
%1997       & 180.8 & 184.2 & 182.7 & 57.72 & 1077   \\ 
%\hline
%1998       & 188.3 & 189.1 & 188.6 & 185.2 & 3086   \\ 
%1999       & 191.4 & 192.1 & 191.6 & 29.53 & 514    \\ 
%1999       & 195.4 & 196.1 & 195.5 & 76.67 & 1137   \\ 
%1999, 2000 & 199.1 & 200.2 & 199.5 & 79.27 & 1090   \\ 
%1999, 2000 & 201.3 & 202.1 & 201.6 & 37.75 & 519    \\ 
%2000       & 202.5 & 205.5 & 204.9 & 82.01 & 1130   \\ 
%2000       & 205.5 & 208.9 & 206.6 & 138.8 & 1717   \\ 
\hline
\end{tabular}
\caption{The used data samplesby year, energy range, mean centre-of-mass-energy, integrated luminosity ${\mathcal L}$.}
\label{tab:luminocity}
\end{table}
}




\newcommand{\TABmontecarlo}{
\begin{table}[htbp]\centering
\begin{tabular}{|c|c|c|c|c|}\hline
%            & Luminocity and process.        &    &    &       \\
Energy      &  Pythia & Herwig & KoralW & KoralW  \\
& $e^+e^-\rightarrow Z\rightarrow \mathrm{hadrons}$ 
& $e^+e^-\rightarrow Z\rightarrow \mathrm{hadrons}$ 
&  $e^+e^-\rightarrow qqqq$ 
&  $e^+e^-\rightarrow qqll$ 
\\\hline\hline
 $130.1$ & & & &  \\
 $130.1$ & & & &  \\
 $136.1$ & & & &  \\
 $136.1$ & & & &  \\
 $161.3$  & & & &  \\
 $172.1$  & & & &  \\
 $182.7$  & & & &  \\
 $189.1$  & & & &  \\
 $192.1$  & & & &  \\
 $195.5$  & & & &  \\
 $199.5$  & & & &  \\
 $204.9$  & & & &  \\
 $206.6$  & & & &  \\
\hline
\end{tabular}
\caption{
A summary on the used samples.
%Ref.~\protect\cite{Zupanc:2013iki}. 
}
\label{tab:montecarlo}
\end{table}
}




\newcommand{\TABalgorithms}{
\begin{table}[htbp]\centering
\begin{tabular}{|c|c|}\hline
            &  Parameters     \\\hline\hline
antikt     &  $R=0.7,P=-1.0$        \\
Durham      &  $R=0.7$               \\
Jade      &      $R=0.7$           \\
Cambridge-Aahchen      &  $R=0.7,P=-1.0$               \\
Siscone      &  $R=0.7,O=0.75$               \\
$ee$-Cambridge      &  $y_{min}={10}^{-10}$               \\
kt      &  $R=0.7$               \\
\hline
\end{tabular}
\caption{
Jet algorithms parameters.
%Ref.~\protect\cite{Zupanc:2013iki}. 
}
\label{tab:algorithms}
\end{table}
}



\newcommand{\TABcutvariations}{
\begin{table}[htbp]\centering
\begin{tabular}{|c|c|c|c|}\hline
Parameters                            &  Reference     & Upper  &  Lower \\\hline\hline
Minimal number of tracks              &      5          & 7  &  \\
Cut on the selection probability  $qqll$    &    $0.50$       & $0.75$  & $0.25$ \\
Cut on the selection probability  $qqqq$    &    $0.25$       &  $0.40$ & $0.1$ \\
$\cos{\theta}$                               &    $0.9$       &   & $0.7$ \\
\hline
\end{tabular}
\caption{
Systematic variations.
%Ref.~\protect\cite{Zupanc:2013iki}. 
}
\label{tab:cutvariations}
\end{table}
}











\newcommand{\FIGexample} {
\begin{tikzpicture}[!htb]
\begin{center}
\includegraphics[width=0.9\linewidth]{../../output/durhamonesixoneJETRtwo}
\caption{The values  }
\label{fig:example}
\end{center}
\end{tikzpicture}
}







\newcommand{\FIGdurhamexample} {
\begin{tikzpicture}[!htb]
\begin{center}
\includegraphics[width=0.9\linewidth]{../../output/durhamonesixoneJETRtwo}
\caption{The two jet rate fraction obtained with Durham algorithm for $\sqrt{s}=161\GeV$}
\label{fig:durhamexample}
\end{center}
\end{tikzpicture}
}


\newcommand{\FIGjadeexample} {
\begin{tikzpicture}[!htb]
\begin{center}
\includegraphics[width=0.9\linewidth]{../../output/jadeonesixoneJETRtwo}
\caption{The two jet rate fraction obtained with Jade algorithm for $\sqrt{s}=161\GeV$}
\label{fig:jadeexample}
\end{center}
\end{tikzpicture}
}



\newcommand{\FIGcambridgeexample} {
\begin{tikzpicture}[!htb]
\begin{center}
\includegraphics[width=0.9\linewidth]{../../output/cambridgeonesixoneJETRfour}
\caption{The four jet rate fraction obtained with Cambridge-Aachen algorithm for $\sqrt{s}=161\GeV$}
\label{fig:cambridgeexample}
\end{center}
\end{tikzpicture}
}



\newcommand{\FIGsisconeexample} {
\begin{tikzpicture}[!htb]
\begin{center}
\includegraphics[width=0.9\linewidth]{../../output/sisconeonesixoneJETRtwo}
\caption{The two jet rate fraction obtained with SiSCone algorithm for $\sqrt{s}=161\GeV$}
\label{fig:sisconeexample}
\end{center}
\end{tikzpicture}
}



\newcommand{\SECtitle}{
\abstract{
A summary of measurements of the fragmentation of charm quarks into a
specific hadron is given. Measurements performed in photoproduction and deep
inelastic scattering in $e^{\pm}p$, $pp$ and $e^+e^-$ collisions
are compared, using up-to-date branching ratios. Within uncertainties, 
all measurements agree, supporting the hypothesis that fragmentation 
is independent of the specific production process. Averages of the 
fragmentation fractions over all measurements are presented.
The average has significantly reduced uncertainties
compared to individual measurements.
}
\makezeustitle 
\newpage}
\newcommand{\eVdist}{\kern-0.06667em}
\newcommand{\GeV}{{\text{Ge}\eVdist\text{V\/}}}
\newcommand{\TeV}{{\text{Te}\eVdist\text{V\/}}}
\newcommand{\Mev}{{\text{Me}\eVdist\text{V\/}}}
\newcommand{\epem}{$e^+e^-\,$}
\newcommand{\ep}{$e^{\pm}p\,$}
\newcommand{\fig}[1]{Fig.~\ref{fig:#1}}
\newcommand{\tab}[1]{Tab.~\ref{tab:#1}}
\newcommand{\sect}[1]{Sec.~\ref{sec:#1}}
\newcommand{\app}[1]{App.~\ref{sec:#1}}
\newcommand{\eq}[1]{Eq.~(\ref{eq:#1})}

\setlength{\textfloatsep}{10pt plus 2.0pt minus 4.0pt}
\begin{document}
\SECtitle
\clearpage
\pagenumbering{arabic}
\pagestyle{plain}
\newpage
%%%%%%%%%%%%%%%%%%%%%%%%%%%%%%%%%%%%%%%%%%%%%%%%%%%%%%%%%%%%%%%%%%%%%%%%
\section{Introduction}
\label{sec:introduction}
The strong interactions  in the Standard Model are described 
with Quantum Chromodynamics(QCD) theory. 
The theory successfully describes 
the interactions between quarks and gluons and is a source for numerous 
predictions. The verification of the QCD predictions of the QCD can
be used for the theory tuning and searches of the physics beyond the 
QCD.
One of QCD predictions that depends strongly on the theory parameters, namely on the constant of strong interaction $\alpha_s$ 
in the topology of the $e^+e^-\rightarrow hadrons$ events, i.e. multi-jet events. One of the reasons why the
topologies of the $e^+e^-\rightarrow hadrons$ events can be predicted with a high
precision is the absence of  strong interaction in the initial state. In the same tame the prediction depends on the 
centre-of-mass energy of the system and the $\alpha_s$.




This paper presents an analysis of the data obtained in $e^+e^-$ 
annihilations to hadrons in the centre-of-mass energy range 
$\sqrt{s}=91-209\GeV$. The data was obtained in the OPAL experiment 
at the LEP storage ring between 1995 and  2001. 
The measured observables are the multiplicities of the jets in the 
hadronic events obtained with various clustering procedures.


%%%%%%%%%%%%%%%%%%%%%%%%%%%%%%%%%%%%%%%%%%%%%%%%%%%%%%%%%%%%%%%%%%%%%%%%
\section{Jet algorithms}
\label{sec:algorithms}
A jet clustering algorithm is a way to simplify the high energy collision event topology 
 and exhibit the underlying physics at the parton level. The main goal of such a procedure is to
 reconstruct the kinematic variables of the particles produced during the primary interaction.
 The energy and the momenta of the partons are reconstructed combining the 
 momenta and energy of the objects originated from the partons into jet objects.
 There are several ways, called jet algorithms,
  to perform the combination. A detailed overview of the properties can be found elsewhere.
 
 
 


%%%%%%%%%%%%%%%%%%%%%%%%%%%%%%%%%%%%%%%%%%%%%%%%%%%%%%%%%%%%%%%%%%%%%%%%
\section{The OPAL detector}
\label{sec:detector}
{\bf Detector description is from~\cite{Abbiendi:2005eq}. Is there a standard one?}
A detailed description of the OPAL detector can be found elsewhere 
\cite{Ahmet:1990eg, Anderson:1997xwa}. Tracking of charged particles was performed by a central
detector, enclosed in a solenoid which provided a uniform axial magnetic field
of 0.435\,T.  The central detector consisted of a two-layer silicon 
microvertex detector, a high precision vertex chamber with both axial
and stereo wire layers, a large volume jet chamber providing both tracking
and ionisation energy loss information,
and additional chambers to measure the $z$ coordinate of tracks as they
left the central detector.\footnote{A right handed coordinate system is used,
with positive $z$ along the $\rm e^-$ beam direction and $x$ pointing 
towards the centre of the LEP ring. The polar and azimuthal angles are denoted
by $\theta$ and $\phi$, and the origin is taken to be the centre of the
detector.} Together these detectors provided tracking coverage for polar angles
$|\cos\theta|<0.96$, with a typical transverse momentum ($p_{\rm T}$) 
resolution\footnote{The convention $c=1$ is used throughout this paper.} of 
$\sigma_{p_{\rm T}}/p_{\rm T}=\sqrt{(0.02)^2+(0.0015p_T)^2}$ 
with $p_{\rm T}$ measured in GeV.
The solenoid coil was surrounded by a time-of-flight counter array and 
a barrel lead-glass electromagnetic calorimeter with a presampler. Including
also the endcap electromagnetic calorimeters, the lead-glass blocks covered
the range $|\cos\theta|<0.98$ with a granularity of about $2.3^\circ$ in both
$\theta$ and $\phi$.
Outside the electromagnetic calorimetry, 
the magnet return yoke was instrumented with
streamer tubes to form a hadronic calorimeter, with angular coverage in the
range $|\cos\theta|<0.91$ and a granularity of about $5^\circ$ in $\theta$
and $7.5^\circ$ in $\phi$. The region $0.91<|\cos\theta|<0.99$ was instrumented
with an additional pole-tip hadronic calorimeter using multi-wire chambers,
having a granularity of about $4^\circ$ in $\theta$ and $11^\circ$ in $\phi$.
The detector was completed
with muon detectors outside the magnet return yoke. These were composed
of drift chambers in the barrel region and limited streamer tubes in the
endcaps, and together covered 93\,\% of the full solid angle.
The integrated luminosity was evaluated using small
angle Bhabha scattering events observed in the forward calorimeters 
\cite{Abbiendi:2003dh}. 
{\bf End of~\cite{Abbiendi:2005eq}}
The 

\section{Data and Monte Carlo samples}
\label{sec:selection}
The analysis is based on the data samples obtained between 1995 and 2001 
at the centre-of-mass energies $\sqrt{s}=91-208\GeV$. A summary on the luminocity
of the used samples is given in Tab.~\ref{tab:luminocity}.
\input{../../output/table_luminocity.tex}
\TABluminocity{The used data samples-by year, energy range, mean centre-of-mass-energy, integrated luminosity }




For the background estimation and detector corrections the Monte Carlo samples 
produced with Pythia6.1, Herwig6.2, Ariadne and, grc4f 2.1  KoralW1.42~\cite{Skrzypek:1995wd} generators 
and passed thought OPAL detector simulation were used.
A short overview of the used Monte Carlo samples is given in Tab.~\ref{tab:montecarlo}.
\TABmontecarlo


\subsection{Event selection}
\label{sec:selection}
The performed event selection is similar to one used in Ref.~\cite{}.

The events where tracking system , electromagnetic calorimeter or the trigger system were inoperational are rejected.
The tracks with small transverse momenta $p_t<0.10\GeV$,{\bf $0.15?$} the minimal  distance to the 
collision axis more than $2cm$ or minimal distance in $z$ direction to the nominal interaction point less that $25cm$
were excluded from the consideration. 
Energy clusters in the electromagnetic calori
meter were required to have energies exceeding $0.10(0.25) \GeV$ in 
the barrel (endcap) region of the detector.


For the further analysis only the
Only events with more than five reconstructed tracks were accepted.

The events were required to have $|\cos\theta|<0.95$, where the  
$\theta|$ is the polar angle of the thrust axis calculated from all track
and cluster objects.

To reject the background event the cuts on the probability of 4-fermion event were applied.
The probability estimation is  calculated in the same way as in Ref.~\cite{Abbiendi:2000eg}.






\subsection{Event reconstruction}
\label{sec:reconstruction}
For the selected events the reconstruction procedure was performed.
The input for the reconstruction procedure for the Monte Carlo generator events
was a set of partons.
The input for the reconstruction procedure on the (simulated)detector level
 was a set of tracks and the electromagnetic calorimeter clusters.
To avoid the double counting of the clustered objects on the detector level the energy-flow 
algorithm~\cite{Ackerstaff:1997nga,Abbiendi:1999sy}, that matched tracks to the 
 clusters in the electromagnetic calorimeter was applied. Therefore each 
 matched pair of track and cluster was replaced with a combined object.
 {\bf I would do so, it is the best solution. But what they did is not clear. Ask Stefan.}

For each event the event shape variables~\cite{OPAL:2011aa} 
were calculated  as  implemented in Rivet package~\cite{Buckley:2010ar}.

The jet rates were obtained with the Durham~\cite{Catani:1991hj}, SiScone~\cite{Cacciari:2005hq}, anti-kt~\cite{Cacciari:2008gp}, Jade~\cite{Bartel:1986ua}
  and Cambridge-Aahen~\cite{Dokshitzer:1997in}
clustering algorithms as implemented in FastJet~\cite{Cacciari:2011ma}.
% for the data and reconstructed events.

The parameters of the clustering procedures are given in Tab.~\ref{tab:algorithms}.
\TABalgorithms

The measured distributions of the jet rates were obtained in the following way.
To take into account the presence of $e^-e^+ \rightarrow llqq$ events in the 
obtained detector level distribution the background subtraction procedure was performed.
The simulated detector level distributions of the $e^-e^+ \rightarrow llqq$ events 
weighted to the same luminocity were subtracted from the detector level distributions.
The obtained distributions were corrected bin by bin to the detector effects. 
The corrections were obtained as a ratio of 
the simulated detector level distributions of the $e^-e^+ \rightarrow llqq$ events 
and the generator level distributions of the same events.
At the next step the  distributions were normalised to the number of weighted entries.

{\bf Not done yet!}
Finally, the distributions were  multiplied bin by bin with the hadronisation corrections.
The later were calculated as a ratio of parton and hadron distributions calculated 
from a large Monte Carlo event sample generated with SHERPA2~\cite{Gleisberg:2008ta}.
{\bf Not done yet End!}


The Figs.~\ref{fig:durhamexample}\ref{fig:jadeexample}\ref{fig:sisconeexample}
present the obtained distributions for the studied clustering algorithms.
\FIGdurhamexample
\FIGjadeexample
%\FIGcambridgeexample
\FIGsisconeexample
The corresponding numerical values can be found in App.~\ref{sec:appendixA}
in Tabs.~\ref{tab:example}. The  presented systematic uncertainties are discussed below.

\FloatBarrier
\section{Systematic uncertainties}
\label{sec:systematic}
It was assumed the
 systematic uncertainties on the obtained results 
 had the several sources.

 To take into account a limited quality of Monte Carlo simulation of the
 QCD processes, the Pythia Monte Carlo simulated event sample used for the detector
 correction was replaced with one produced with Herwig.
 The uncertainty of the background modelling was estimated by varying the 
 background amount by $\pm5\%$.
To take into account a limited efficiently of the energy-flow 
algorithm the same reconstruction was performed without it.
 To take into account the influence of selection procedure
the background rejection cuts described in Sec.~\ref{sec:selection}. % independantly.
 The actual values of the independently performed variations are shown in Tab.~\ref{tab:cutvariations}.
 \TABcutvariations
 
The differences between the reference distributions and the distributions obtained with 
the variation procedure were bin-by-bin added in quadratures resulting the systematic uncertainty for 
every bin of the reference distribution.
 
% For each quantity  the  pair of cut changes to lower and higher value considering as one variation.


%\TABGdurhamonesixoneJETRthree{$y$ cut}{Durham 161 JETR3}

%\TABGdurhamoneseventwoJETRthree{$y$ cut}{Durham 172 JETR3}
%\FIGdurhamonesixoneJETRtwo{Durham 161 JETR2}
%\FIGdurhamoneseventwoJETRfour{Durham 172 JETR3}
%\FIGdurhamoneseventwoJETRsix{Durham 172 JETR6}

%\FIGdurhamoneseventwoT{Durham 172 $T$}
%\FIGdurhamoneeightnineJETRfour{Durham 189 $T$}




%%%%%%%%%%%%%%%%%%%%%%%%%%%%%%%%%%%%%%%%%%%%%%%%%%%%%%%%%%%%%%%%%%%%%%%%
\section{Comparison to the previous measurements}                      
\label{sec:comparison}
The previous measurements of the jet rates were performed on the LEP by 
the OPAL, DELPHI, ALEPH and L3 experiments~\cite{Alexander:1996kh,
Ackerstaff:1997kk,Acton:1992fa,Akrawy:1989rg,Heister:2003aj,Abdallah:2003xz,Achard:2004sv}.





%%%%%%%%%%%%%%%%%%%%%%%%%%%%%%%%%%%%%%%%%%%%%%%%%%%%%%%%%%%%%%%%%%%%%%%%
\section{Summary}                      
\label{sec:summary}
The jet rate observables  in $e^+e^-$ 
annihilations to hadrons were measured with various clustering procedures.
The new results obtained with 
clustering algorithms  used earlier agree with the previously published measurements.
The results obtained with the recently developed clustering algorithms  
provide an input for the validation of QCD and precise estimation of its parameters.
%
%%%%%%%%%%%%%%%%%%%%%%%%%%%%%%%%%%%%%%%%%%%%%%%%%%%%%%%%%%%%%%%%%%%%%%%%
\section*{Acknowledgements}
\label{sec:acknowledgements}
We thank 
\FloatBarrier
\begin{appendices}
%%%%%%%%%%%%%%%%%%%%%%%%%%%%%%%%%%%%%%%%%%%%%%%%%%%%%%%%%%%%%%%%%%%%%%%%
\section{AppendixA}
\label{sec:appendixA}
%\TABexample
%\TABexample
%\TABexample
%\TABexample
%\TABexample
\input{../../output/tables_161.tex}
\TABGdurhamonesixoneJETRthree{$y$ cut}{Durham 161 JETR3}

\end{appendices}

{\bibliographystyle{./opalJRT}{\raggedright\bibliography{opalJRT.bib}}}\vfill\eject
\clearpage
\end{document}
